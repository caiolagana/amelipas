\documentclass[12pt]{article}
\usepackage[brazil]{babel}
\usepackage[utf8]{inputenc}
\usepackage{geometry}
\usepackage{enumitem}
\geometry{a4paper, margin=2.2cm}
%%% Times New Roman - Traditional legal document font
\usepackage{newtxtext,newtxmath}
\linespread{1.0}


\newcommand{\capitulo}[1]{\begin{center}\fontseries{b}\selectfont\textbf{\MakeUppercase{#1}}\end{center}}
\newcommand{\artigo}[1]{\vspace{1.0em}\noindent\textbf{#1}\hspace{0.75em}}
\newcommand{\paragrafo}[1]{\vspace{1.0em}\noindent{#1}\hspace{0.75em}}
\newcommand{\titulo}[1]{\begin{center}\fontsize{18}{22}\fontseries{b}\selectfont{#1\\[1.0em]}\end{center}}
\newcommand{\subtitulo}[1]{\begin{center}\fontsize{16}{17}\fontseries{m}\selectfont{#1\\[0.5em]}\end{center}}
\newcommand{\subsubtitulo}[1]{\begin{center}\fontsize{10}{12}\fontseries{m}\selectfont{#1\\[3.0em]}\end{center}}


\begin{document}


\titulo{CONTRATO DE REVITALIZAÇÃO DE ÁREA}
\subtitulo{Portal da Estrada Parque do Rosário}


\artigo{CLÁUSULA 1 – PARTES} Este contrato é celebrado entre, AURELINO SOBRENOME doravante denominado PROPRIETÁRIO, residente e domiciliado à [endereço completo], portador do RG nº [número] e CPF nº [número]; e os moradores e interessados abaixo assinados, doravante denominados INTERESSADOS, residentes ou transeuntes à Estrada do Rosário, conforme lista de assinaturas ao final deste documento.

\artigo{CLÁUSULA 2 – OBJETO} Este contrato tem por objeto a permissão, por parte do PROPRIETÁRIO, à revitalização de uma área de terra situada em barranco às margens da Estrada do Rosário, doravante denominada “ÁREA”, mediante o plantio de mudas de plantas nativas.

\artigo{CLÁUSULA 3 – PLANTAS PERMITIDAS} Fica permitido o plantio, exclusivamente, das seguintes espécies nativas:

\begin{enumerate}[label=\alph*), itemsep=0pt, parsep=0pt]
  \item Ipê-amarelo (Handroanthus albus)
  \item Pitanga (Eugenia uniflora)
  \item Quaresmeira (Tibouchina granulosa)
  \item Jabuticabeira (Plinia cauliflora)
  \item Macela (Achyrocline satureioides)
  \item Manacá-da-serra (Tibouchina mutabilis)
\end{enumerate}

\paragrafo{§ ÚNICO} Outras espécies nativas da região poderão ser autorizadas desde que haja concordância expressa do PROPRIETÁRIO.

\artigo{CLÁUSULA 4 – OBRIGAÇÕES DAS PARTES} Ficam estabelecidas as seguintes obrigações:

\paragrafo{Do Proprietário:}
\begin{enumerate}[label=\alph*), itemsep=0pt, parsep=0pt]
\item Permitir o acesso à ÁREA para fins de revitalização e plantio;  
\item Não realizar interferências que prejudiquem o projeto de revitalização, salvo por motivos de força maior ou necessidade legal.
\end{enumerate}

\paragrafo{Dos Moradores e Interessados:}
\begin{enumerate}[label=\alph*), itemsep=0pt, parsep=0pt]
\item Responsabilizar-se pela aquisição das mudas das espécies listadas na CLÁUSULA 3, bem como por todos os custos relacionados ao plantio e manutenção das mesmas;
\item Realizar o plantio, rega, limpeza e demais atividades de manutenção necessárias para o bom desenvolvimento das plantas
\item Realizar, ao menos uma vez ao ano, o manejo adequado da área, com remoção de espécies invasoras e poda das espécies introduzidas, de forma a promover o crescimento adequado das mesmas;
\item Zelar pela conservação da ÁREA, evitando qualquer tipo de dano, poda ou corte irregular das plantas introduzidas.
\end{enumerate}

\paragrafo{§ ÚNICO} Os INTERESSADOS poderão contratar um terceiro para a realização dos serviços de manejo, mediante consentimento do PROPRIETÁRIO em relação à pessoa contratada. Todo custo decorrente dessa contratação será de responsabilidade exclusiva dos INTERESSADOS.

\artigo{CLÁUSULA 5 – RESCISÃO E PERDA DE VALIDADE} O presente contrato perderá sua validade caso os INTERESSADOS venham a descumprir qualquer uma das obrigações estabelecidas neste instrumento.

\paragrafo{§ ÚNICO} Em caso de descumprimento, pelo PROPRIETÁRIO, das obrigações que lhe competem nos termos deste instrumento, ficará a seu encargo a responsabilização integral pela revitalização e recuperação da ÁREA, a qual se encontra configurada como Área de Preservação Permanente (APP), nos termos do Art. 4º da Lei nº 12.651/2012, por tratar-se de encosta com declividade superior a 45 graus.


\artigo{CLÁUSULA 6 – DISPOSIÇÕES FINAIS}
\begin{enumerate}[label=\alph*), itemsep=0pt, parsep=0pt]
\item Este contrato tem prazo indeterminado de validade;
\item Qualquer alteração deverá ser feita por escrito e assinada por ambas as partes.
\end{enumerate}



% ---

% [Local], [Data].

% **Assinaturas:**

% ________________________________  
% Proprietário

% ________________________________  
% Moradores/Interessados

% (Nome, RG, Assinatura de cada interessado)

\vspace{2em}

\begin{flushright}
Delfim Moreira, XX de Novembro de 2025\vspace{2em}
\rule{10cm}{0.5pt}\\Aurelino Sobrenome\\[2em]
\rule{10cm}{0.5pt}\\Interessado 1\\[2em]
\rule{10cm}{0.5pt}\\Interessado 2\\[2em]
\rule{10cm}{0.5pt}\\Interessado 3\\[2em]
\rule{10cm}{0.5pt}\\Interessado 4\\[2em]
\rule{10cm}{0.5pt}\\Interessado 5\\[2em]
\rule{10cm}{0.5pt}\\Interessado 6\\[2em]
\rule{10cm}{0.5pt}\\Interessado 7\\[2em]
\end{flushright}

\end{document}


