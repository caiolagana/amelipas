\documentclass[12pt]{article}
\usepackage[brazil]{babel}
\usepackage[utf8]{inputenc}
\usepackage{geometry}
\usepackage{enumitem}
\usepackage{url}
\geometry{a4paper, margin=2.0cm}
%%% Times New Roman - Traditional legal document font
\usepackage{newtxtext,newtxmath}
\linespread{1.0}


\newcommand{\capitulo}[1]{\vspace{1.0em}\begin{center}\fontseries{b}\selectfont\textbf{\MakeUppercase{#1}}\end{center}}
\newcommand{\artigo}[1]{\vspace{1.0em}\noindent\textbf{#1}\hspace{0.75em}}
\newcommand{\paragrafo}[1]{\vspace{1.0em}\noindent{#1}\hspace{0.75em}}
\newcommand{\titulo}[1]{\begin{center}\fontsize{18}{22}\fontseries{b}\selectfont{#1\\[1.0em]}\end{center}}
\newcommand{\subtitulo}[1]{\begin{center}\fontsize{16}{17}\fontseries{m}\selectfont{#1\\[0.5em]}\end{center}}
\newcommand{\subsubtitulo}[1]{\begin{center}\fontsize{10}{12}\fontseries{m}\selectfont{#1\\[2.0em]}\end{center}}


\begin{document}


\titulo{ESTATUTO SOCIAL}
\subtitulo{Associação de Meliponicultores para Recuperação \\e Preservação Ambiental da Serra da Mantiqueira}\subsubtitulo{CNPJ 12.3456.789/0001-00}


\capitulo{Capitulo I - Nome, Sede, Fins e Duração}

\artigo{Artigo 1º} A Associação de Meliponicultores para Recuperação e Preservação Ambiental da Serra da Mantiqueira (de abreviatura AMELIPAS) é uma associação sem fins lucrativos, de prazo indeterminado de duração, com sede e foro na cidade de Delfim Moreira, Estado de Minas Gerais, na [Endereço completo] e endereço eletrônico \url{www.amelipas.org}

\artigo{Artigo 2º} A Associação tem por finalidade promover e fomentar o desenvolvimento ecológico sustentável, a conservação e recuperação ambiental e a capacitação agroecológica dos associados e comunidade, por meio das seguintes atividades:
\begin{enumerate}[label=\alph*)]
  \item Fomentar a meliponicultura (criação de abelhas nativas sem ferrão) como ferramenta de conservação e recuperação ambiental e geração de renda, de acordo com as diretrizes estabelecidas na Lei nº 14.639/2023;
  \item Desenvolver e implementar programas de Pagamento por Serviços Ambientais (PSA), conforme Lei nº 14.119/2021, com ênfase na proteção e recuperação de nascentes, áreas degradadas, e demais áreas de interesse social caracterizadas pela Lei nº 12.651/2012;
  \item Oferecer cursos e palestras de educação ambiental, focados na flora nativa e seus polinizadores, visando a conscientização ecológica, o manejo sustentável de pastagens, capoeiras e áreas de mata, tendo como público alvo a comunidade e as escolas;
  \item Criar e manter um viveiro de mudas nativas e banco de sementes para apoiar ações de reflorestamento e recuperação ecológica;
  \item Adquirir e operar maquinário (tratores, escavadeiras, minitratores) para a execução de serviços de conservação do solo e retenção de água, particularmente a construção de terraços em curva de nível, açudes e barragens em áreas estratégicas.
\end{enumerate}

\paragrafo{§ 1º} Para consecução de suas atividades, a Associação irá concorrer a editais e processos seletivos públicos e privados para a captação de recursos destinados a projetos de reflorestamento e serviços ambientais.


\paragrafo{§ 2º} A Associação poderá firmar parcerias com outras instituições, públicas ou privadas, visando o fortalecimento de suas ações e a ampliação de seus resultados.


\capitulo{Capitulo II - Definições e ditrizes}

\artigo{§ 4º} Para os efeitos deste Estatuto, entende-se por:
\begin{enumerate}[label=\Roman* -]
  \item Pasto de abelhas: área de vegetação nativa ou cultivada, destinada à alimentação de abelhas nativas sem ferrão.
  \item PREX: valor a ser pago por cada décimo de hectare recuperado (0,1 ha), conforme definido no regimento interno da Associação.
  \item Manejo sustentável: conjunto de práticas que visam a conservação e recuperação ambiental, promovendo o equilíbrio ecológico e a sustentabilidade dos recursos naturais.
\end{enumerate}

\paragrafo{§ 3º} O pasto para abelhas é conceito central em torno do qual os programas de recuperação ambiental da Associação serão estruturados, uma vez que promove a geração de renda dos meliponicultores em sinergia com a conservação da flora nativa e a sustentabilidade dos ecossistemas locais.

\paragrafo{§ 3º} A formação do banco de sementes de espécies nativas realizar-se-á, sempre que possível, adquirindo-as da comunidade local, de forma a estimular a criação deste nicho de mercado e priorizar a diversidade genética local.

\paragrafo{§ 4º} O valor de PSA do hectare recuperado será publicado no regimento interno da Associação, a será revisto anualmente conforme índice oficial de inflação.

% \capitulo{Capítulo II - Dos Associados}
% \artigo{Artigo 3º} - A associação é composta por número ilimitado de associados, distribuídos nas seguintes categorias: [fundadores, efetivos, honorários, etc].

% \artigo{Artigo 4º} - São direitos dos associados: [listar direitos].

% \artigo{Artigo 5º} - São deveres dos associados: [listar deveres].

% \capitulo{Capítulo III - Da Administração}
% \artigo{Artigo 6º} - A associação será administrada por uma Diretoria composta por: [Presidente, Vice-Presidente, etc].

% \artigo{Artigo 7º} - Compete à Diretoria: [listar competências].

% \capitulo{Capítulo IV - Das Disposições Gerais}
% \artigo{Artigo 8º} - O presente estatuto só poderá ser alterado em Assembleia Geral Extraordinária, especialmente convocada para esse fim.

% \artigo{Artigo 9º} - Os casos omissos serão resolvidos pela Diretoria, ad referendum da Assembleia Geral.

% \vspace{1cm}
% \begin{flushright}
% [Local], [Data]
% \end{flushright}

\end{document}