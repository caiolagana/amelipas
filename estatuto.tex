\documentclass[12pt]{article}
\usepackage[brazil]{babel}
\usepackage[utf8]{inputenc}
\usepackage{geometry}
\usepackage{enumitem}
\geometry{a4paper, margin=2.0cm}
%%% Times New Roman - Traditional legal document font
\usepackage{newtxtext,newtxmath}
\linespread{1.0}


\newcommand{\capitulo}[1]{\begin{center}\fontseries{b}\selectfont\textbf{\MakeUppercase{#1}}\end{center}}
\newcommand{\artigo}[1]{\vspace{1.0em}\noindent\textbf{#1}}
\newcommand{\paragrafo}[1]{\vspace{1.0em}\noindent{#1}}
\newcommand{\titulo}[1]{\begin{center}\fontsize{18}{22}\fontseries{b}\selectfont{#1\\[1.0em]}\end{center}}
\newcommand{\subtitulo}[1]{\begin{center}\fontsize{16}{17}\fontseries{m}\selectfont{#1\\[0.5em]}\end{center}}
\newcommand{\subsubtitulo}[1]{\begin{center}\fontsize{10}{12}\fontseries{m}\selectfont{#1\\[3.0em]}\end{center}}


\begin{document}


\titulo{ESTATUTO SOCIAL}
\subtitulo{Associação de Meliponicultores para Recuperação \\e Preservação Ambiental da Serra da Mantiqueira}\subsubtitulo{CNPJ 12.3456.789/0001-00}


\capitulo{Capitulo I - Nome, Sede, Fins e Duração}

\artigo{Artigo 1º} A Associação de Meliponicultores para Recuperação e Preservação Ambiental da Serra da Mantiqueira (de abreviatura AMELIPAS) é uma associação sem fins lucrativos, de prazo indeterminado de duração, com sede e foro na cidade de Delfim Moreira, Estado de Minas Gerais, na [Endereço completo].

\artigo{Artigo 2º} - A Associação tem por finalidade promover e fomentar o desenvolvimento sustentável, a conservação e recuperação ambiental e a capacitação agroecológica dos associados, por meio das seguintes atividades:
\begin{enumerate}[label=\alph*)]
  \item Fomentar a meliponicultura (criação de abelhas nativas sem ferrão) como ferramenta de conservação e recuperação ambiental e geração de renda.
  \item Desenvolver e implementar programas de Pagamento por Serviços Ambientais, instituido pela lei 14.119/2021, com ênfase na proteção e recuperação de nascentes, áreas degradadas de elevado potencial biodiverso, e demais áreas de interesse social caracterizadas pela lei 12.651/2012.
  \item Elaborar, apoiar e executar projetos de educação ambiental junto a escolas e à comunidade, focados na flora nativa, polinizadores e ecossistemas locais.
  \item Oferecer cursos e consultoria técnica especializada aos associados e à comunidade, visando o manejo sustentável de pastagens e conservação do solo.
  \item Criar e manter um banco de sementes de espécies nativas para subsidiar ações de reflorestamento e recuperação ecológica.
  \item Adquirir e operar maquinário (tratores, escavadeiras, minitratores) para a execução de serviços de conservação do solo e retenção de água, incluindo a construção de terraços em curva de nível em áreas estratégicas, enriquecendo assim o abastecimento dos lençóis freáticos e diminuindo as enxurradas.
\end{enumerate}

\paragrafo{§ PRIMEIRO} Para consecução de suas atividades, a Associação irá concorrer a editais e processos seletivos públicos e privados para a captação de recursos destinados a projetos de reflorestamento e serviços ambientais.

\paragrafo{§ SEGUNDO} A Associação poderá firmar parcerias com outras instituições, públicas ou privadas, visando o fortalecimento de suas ações e a ampliação de seus resultados.

% \capitulo{Capítulo II - Dos Associados}
% \artigo{Artigo 3º} - A associação é composta por número ilimitado de associados, distribuídos nas seguintes categorias: [fundadores, efetivos, honorários, etc].

% \artigo{Artigo 4º} - São direitos dos associados: [listar direitos].

% \artigo{Artigo 5º} - São deveres dos associados: [listar deveres].

% \capitulo{Capítulo III - Da Administração}
% \artigo{Artigo 6º} - A associação será administrada por uma Diretoria composta por: [Presidente, Vice-Presidente, etc].

% \artigo{Artigo 7º} - Compete à Diretoria: [listar competências].

% \capitulo{Capítulo IV - Das Disposições Gerais}
% \artigo{Artigo 8º} - O presente estatuto só poderá ser alterado em Assembleia Geral Extraordinária, especialmente convocada para esse fim.

% \artigo{Artigo 9º} - Os casos omissos serão resolvidos pela Diretoria, ad referendum da Assembleia Geral.

% \vspace{1cm}
% \begin{flushright}
% [Local], [Data]
% \end{flushright}

\end{document}