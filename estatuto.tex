\documentclass[11pt]{article}
\usepackage[brazil]{babel}
\usepackage[utf8]{inputenc}
\usepackage{geometry}
\geometry{a4paper, margin=1.5cm}
%%% Combo: Palatino + Helvetica + Courier, com EulerVM para matemática
\usepackage{mathpazo}
\usepackage[scaled=.95]{helvet}
\usepackage{courier}
\linespread{1.05} %%% Palatino needs more spacing.

% Comando para section centralizada e maiúscula
\newcommand{\estatutoSection}[1]{\begin{center}\textbf{\MakeUppercase{#1}}\end{center}}

% Comando para formatar artigos
\newcommand{\artigo}[1]{\vspace{1.0em}\noindent\textbf{#1}}

\title{Associação de Meliponicultores para Recuperação e Preservação Ambiental da Serra da Mantiqueira\\[1.0em] ESTATUTO SOCIAL}
\date{}

\begin{document}

\maketitle

\estatutoSection{Capitulo I - Nome, Sede, Objeto e Duração}

\artigo{Artigo 1º} A associação se denominará Associação de Meliponicultores para Recuperação e Preservação Ambiental da Serra da Mantiqueira (Abreviatura AMELIPAS).

\artigo{Artigo 2º} - A Associação terá sua sede e foro na cidade de Delfim Moreira, Estado de Minas Gerais, na [Endereço completo].

\artigo{Artigo 3º} - A Associação tem por finalidade principal atuar sem fins lucrativos, visando promover e fomentar o desenvolvimento sustentável, a conservação ambiental e a capacitação dos associados, por meio das seguintes atividades:

\begin{enumerate}
  \item Fomentar a meliponicultura (criação de abelhas nativas sem ferrão) como ferramenta de conservação e recuperação ambiental e geração de renda.
  \item Desenvolver e implementar programas de Pagamento por Serviços Ambientais (PSA), com ênfase na recuperação de áreas degradadas e na proteção de recursos hídricos.
  \item Criar e manter um banco de sementes de espécies nativas para subsidiar ações de reflorestamento e recuperação ecológica.
  \item Elaborar, apoiar e executar projetos de educação ambiental, focados na flora nativa, polinizadores e ecossistemas locais.
  \item Oferecer consultoria técnica especializada aos associados, visando o manejo sustentável de pastagens e conservação do solo.
  \item Adquirir e operar tratores e minitratores para a execução de serviços de conservação do solo e retenção de água, incluindo a construção de terraços em curva de nível.
  \item Participar de editais e processos seletivos públicos e privados para a captação de recursos destinados a projetos de reflorestamento e serviços ambientais.
  \item Gerir e comercializar créditos de carbono, bem como outros ativos ambientais, oriundos das atividades de conservação e restauração da associação.
\end{enumerate}

\artigo{Artigo 4º} - O prazo de duração da Associação é indeterminado.

% \estatutoSection{Capítulo II - Dos Associados}
% \artigo{Artigo 3º} - A associação é composta por número ilimitado de associados, distribuídos nas seguintes categorias: [fundadores, efetivos, honorários, etc].

% \artigo{Artigo 4º} - São direitos dos associados: [listar direitos].

% \artigo{Artigo 5º} - São deveres dos associados: [listar deveres].

% \estatutoSection{Capítulo III - Da Administração}
% \artigo{Artigo 6º} - A associação será administrada por uma Diretoria composta por: [Presidente, Vice-Presidente, etc].

% \artigo{Artigo 7º} - Compete à Diretoria: [listar competências].

% \estatutoSection{Capítulo IV - Das Disposições Gerais}
% \artigo{Artigo 8º} - O presente estatuto só poderá ser alterado em Assembleia Geral Extraordinária, especialmente convocada para esse fim.

% \artigo{Artigo 9º} - Os casos omissos serão resolvidos pela Diretoria, ad referendum da Assembleia Geral.

% \vspace{1cm}
% \begin{flushright}
% [Local], [Data]
% \end{flushright}

\end{document}